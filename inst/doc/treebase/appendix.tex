\documentclass[author-year, review, 12pt]{elsarticle} %review=doublespace preprint=single 5p=2 column
\usepackage{amsmath, amsfonts, amssymb}  % extended mathematics
\usepackage{ifxetex}
\ifxetex
  \usepackage{fontspec,xltxtra,xunicode}
  \defaultfontfeatures{Mapping=tex-text,Scale=MatchLowercase}
  \setmainfont[Mapping=tex-text]{TeX Gyre Pagella}
  \setsansfont[Mapping=tex-text]{DejaVu Sans}
  \setmonofont{Bitstream Vera Sans Mono}
\else
  \usepackage[mathletters]{ucs}
  \usepackage[utf8x]{inputenc}
\fi

% My package additions
\usepackage[hyphens]{url}
\usepackage{lineno} % add 
\linenumbers % turns line numbering on 
\bibliographystyle{elsarticle-harv}
\biboptions{sort&compress} % For natbib
\usepackage{graphicx}
\usepackage{booktabs} % book-quality tables

%% Redefines the elsarticle footer
\makeatletter
\def\ps@pprintTitle{%
 \let\@oddhead\@empty
 \let\@evenhead\@empty
 \def\@oddfoot{\it \hfill\today}%
 \let\@evenfoot\@oddfoot}
\makeatother

% A modified page layout
\textwidth 6.75in
\oddsidemargin -0.15in
\evensidemargin -0.15in
\textheight 9in
\topmargin -0.5in


\usepackage{microtype}
%\usepackage{fancyhdr}
%\pagestyle{fancy}
%\pagenumbering{arabic}

\usepackage{listings}
\lstnewenvironment{code}{\lstset{language=Haskell,basicstyle=\small\ttfamily}}{}


\setlength{\parindent}{0pt}
\setlength{\parskip}{6pt plus 2pt minus 1pt}


%%% Syntax Highlighting for code  %%%
%%% Adapted from knitr book %%% 
\usepackage{fancyvrb}
\DefineVerbatimEnvironment{Highlighting}{Verbatim}{commandchars=\\\{\}}
% Add ',fontsize=\small' for more characters per line
\newenvironment{Shaded}{}{}
\newcommand{\KeywordTok}[1]{\textcolor[rgb]{0.00,0.44,0.13}{\textbf{{#1}}}}
\newcommand{\DataTypeTok}[1]{\textcolor[rgb]{0.56,0.13,0.00}{{#1}}}
\newcommand{\DecValTok}[1]{\textcolor[rgb]{0.25,0.63,0.44}{{#1}}}
\newcommand{\BaseNTok}[1]{\textcolor[rgb]{0.25,0.63,0.44}{{#1}}}
\newcommand{\FloatTok}[1]{\textcolor[rgb]{0.25,0.63,0.44}{{#1}}}
\newcommand{\CharTok}[1]{\textcolor[rgb]{0.25,0.44,0.63}{{#1}}}
\newcommand{\StringTok}[1]{\textcolor[rgb]{0.25,0.44,0.63}{{#1}}}
\newcommand{\CommentTok}[1]{\textcolor[rgb]{0.38,0.63,0.69}{\textit{{#1}}}}
\newcommand{\OtherTok}[1]{\textcolor[rgb]{0.00,0.44,0.13}{{#1}}}
\newcommand{\AlertTok}[1]{\textcolor[rgb]{1.00,0.00,0.00}{\textbf{{#1}}}}
\newcommand{\FunctionTok}[1]{\textcolor[rgb]{0.02,0.16,0.49}{{#1}}}
\newcommand{\RegionMarkerTok}[1]{{#1}}
\newcommand{\ErrorTok}[1]{\textcolor[rgb]{1.00,0.00,0.00}{\textbf{{#1}}}}
\newcommand{\NormalTok}[1]{{#1}}
\usepackage{enumerate}
\usepackage{ctable}
\usepackage{float}

% This is needed because raggedright in table elements redefines \\:
\newcommand{\PreserveBackslash}[1]{\let\temp=\\#1\let\\=\temp}
\let\PBS=\PreserveBackslash
\usepackage[normalem]{ulem}
\newcommand{\textsubscr}[1]{\ensuremath{_{\scriptsize\textrm{#1}}}}

% Configure hyperlinks package
\usepackage[breaklinks=true,linktocpage,pdftitle={},pdfauthor={},colorlinks]{hyperref}
\hypersetup{breaklinks=true, pdfborder={0 0 0}}

% Pandoc toggle for numbering sections (defaults to be off)
\setcounter{secnumdepth}{0}


\VerbatimFootnotes % allows verbatim text in footnotes

% Pandoc header



\begin{document}
\begin{frontmatter}
  \title{}
  \author[cpb]{Carl Boettiger\corref{cor1}}
  \ead{cboettig@ucdavis.edu}
  \cortext[cor1]{Corresponding author}
  \address[cpb]{Center for Population Biology, University of California, Davis, California 95616}
  \author[stats]{Duncan Temple Lang}
  \address[stats]{Department of Statistics, University of California, Davis, California 95616}
 \end{frontmatter}


\section{Appendix}

\subsection{Reproducible computation: A diversification rate analysis}

This appendix illustrates the diversification rate analysis discussed in
the main text. For completeness we begin by executing the code discussed
in the manuscript which locates, downloads, and imports the relevant
data:

Different diversification models make different assumptions about the
rate of speciation, extinction, and how these rates may be changing over
time. The original authors consider eight different models, implemented
in the laser package (Rabosky 2006). This code fits each of the eight
models to that data:

\begin{Shaded}
\begin{Highlighting}[]
\KeywordTok{library}\NormalTok{(ape)}
\KeywordTok{library}\NormalTok{(laser)}
\NormalTok{models <- }\KeywordTok{list}\NormalTok{(}
  \DataTypeTok{yule =} \KeywordTok{pureBirth}\NormalTok{(bt),  }
  \DataTypeTok{birth_death =} \KeywordTok{bd}\NormalTok{(bt),     }
  \DataTypeTok{yule.2.rate =} \KeywordTok{yule2rate}\NormalTok{(bt),}
  \DataTypeTok{linear.diversity.dependent =} \KeywordTok{DDL}\NormalTok{(bt),    }
  \DataTypeTok{exponential.diversity.dependent =} \KeywordTok{DDX}\NormalTok{(bt),}
  \DataTypeTok{varying.speciation_rate =} \KeywordTok{fitSPVAR}\NormalTok{(bt),  }
  \DataTypeTok{varying.extinction_rate =} \KeywordTok{fitEXVAR}\NormalTok{(bt),  }
  \DataTypeTok{varying_both =} \KeywordTok{fitBOTHVAR}\NormalTok{(bt))}
\end{Highlighting}
\end{Shaded}
Each of the model estimate includes an Akaike Information Criterion
(AIC) score indicating the goodness of fit, penalized by model
complexity (lower scores indicate better fits) We ask R to tell us which
model has the lowest AIC score,

\begin{Shaded}
\begin{Highlighting}[]
\NormalTok{aics <- }\KeywordTok{sapply}\NormalTok{(models, function(model) model$aic)}
\NormalTok{best_fit <- }\KeywordTok{names}\NormalTok{(models[}\KeywordTok{which.min}\NormalTok{(aics)])}
\end{Highlighting}
\end{Shaded}
and confirm the result presented in E. P. Derryberry et al. (2011); that
the best-fit model in the laser analysis was a Yule (net diversification
rate) model with two separate rates.

We can ask \texttt{TreePar} to see if a model with more rate shifts is
favoured over this single shift, a question that was not possible to
address using the tools provided in \texttt{laser}. The previous
analysis also considers a birth-death model that allowed speciation and
extinction rates to be estimated separately, but did not allow for a
shift in the rate of such a model. In the main text we introduced a
model from Stadler (2011) that permitted up to 3 change-points in the
speciation rate of the Yule model,

\begin{Shaded}
\begin{Highlighting}[]
\NormalTok{yule_models <- }\KeywordTok{bd.shifts.optim}\NormalTok{(x, }\DataTypeTok{sampling =} \KeywordTok{c}\NormalTok{(}\DecValTok{1}\NormalTok{,}\DecValTok{1}\NormalTok{,}\DecValTok{1}\NormalTok{,}\DecValTok{1}\NormalTok{), }
  \DataTypeTok{grid =} \DecValTok{5}\NormalTok{, }\DataTypeTok{start =} \DecValTok{0}\NormalTok{, }\DataTypeTok{end =} \DecValTok{60}\NormalTok{, }\DataTypeTok{yule =} \OtherTok{TRUE}\NormalTok{)[[}\DecValTok{2}\NormalTok{]]}
\end{Highlighting}
\end{Shaded}
We can also compare the performance of models which allow up to three
shifts while estimating extinction and speciation rates separately:

\begin{Shaded}
\begin{Highlighting}[]
\NormalTok{birth_death_models <- }\KeywordTok{bd.shifts.optim}\NormalTok{(x, }\DataTypeTok{sampling =} \KeywordTok{c}\NormalTok{(}\DecValTok{1}\NormalTok{,}\DecValTok{1}\NormalTok{,}\DecValTok{1}\NormalTok{,}\DecValTok{1}\NormalTok{), }
  \DataTypeTok{grid =} \DecValTok{5}\NormalTok{, }\DataTypeTok{start =} \DecValTok{0}\NormalTok{, }\DataTypeTok{end =} \DecValTok{60}\NormalTok{, }\DataTypeTok{yule =} \OtherTok{FALSE}\NormalTok{)[[}\DecValTok{2}\NormalTok{]]}
\end{Highlighting}
\end{Shaded}
The models output by these functions are ordered by increasing number of
shifts. We can select the best-fitting model by AIC score, which is
slightly cumbersome in \texttt{TreePar} syntax. First, we compute the
AIC scores of both the \texttt{yule\_models} and the
\texttt{birth\_death\_models} we fitted above,

\begin{Shaded}
\begin{Highlighting}[]
\NormalTok{yule_aic <- }
\KeywordTok{sapply}\NormalTok{(yule_models, function(pars)}
                    \DecValTok{2} \NormalTok{* (}\KeywordTok{length}\NormalTok{(pars) - }\DecValTok{1}\NormalTok{) + }\DecValTok{2} \NormalTok{* pars[}\DecValTok{1}\NormalTok{] )}
\NormalTok{birth_death_aic <- }
\KeywordTok{sapply}\NormalTok{(birth_death_models, function(pars)}
                            \DecValTok{2} \NormalTok{* (}\KeywordTok{length}\NormalTok{(pars) - }\DecValTok{1}\NormalTok{) + }\DecValTok{2} \NormalTok{* pars[}\DecValTok{1}\NormalTok{] )}
\end{Highlighting}
\end{Shaded}
Then we generate a list identifying which model has the best (lowest)
AIC score among the Yule models and which has the best AIC score among
the birth-death models,

\begin{Shaded}
\begin{Highlighting}[]
\NormalTok{best_no_of_rates <- }\KeywordTok{list}\NormalTok{(}\DataTypeTok{Yule =} \KeywordTok{which.min}\NormalTok{(yule_aic), }
                         \DataTypeTok{birth.death =} \KeywordTok{which.min}\NormalTok{(birth_death_aic))}
\end{Highlighting}
\end{Shaded}
The best model is then whichever of these has the smaller AIC value.

\begin{Shaded}
\begin{Highlighting}[]
\NormalTok{best_model <- }\KeywordTok{which.min}\NormalTok{(}\KeywordTok{c}\NormalTok{(}\KeywordTok{min}\NormalTok{(yule_aic), }\KeywordTok{min}\NormalTok{(birth_death_aic)))}
\end{Highlighting}
\end{Shaded}
which still confirms that the Yule 2-rate model is still the best choice
based on AIC score. Of the eight models in this second analysis, only
three were in the original set considered (Yule 1-rate and 2-rate, and
birth-death without a shift), so we could by no means have been sure
ahead of time that a birth death with a shift, or a Yule model with a
greater number of shifts, would not have fitted better.

\section{References}

Derryberry, Elizabeth P., Santiago Claramunt, Graham Derryberry, R.
Terry Chesser, Joel Cracraft, Alexandre Aleixo, Jorge Pérez-Emán, J. V.
Remsen Jr, and Robb T. Brumfield. 2011. ``Lineage diversification and
morphological evolution in a large-scale continental radiation: the
neotropical ovenbirds and woodcreepers (Aves: Furnariidae).''
\emph{Evolution} (Jul). doi:10.1111/j.1558-5646.2011.01374.x.

Rabosky, Daniel L. 2006. ``LASER: a maximum likelihood toolkit for
detecting temporal shifts in diversification rates from molecular
phylogenies.'' \emph{Evolutionary bioinformatics online} 2 (Jan):
273--6.

Stadler, Tanja. 2011. ``Mammalian phylogeny reveals recent
diversification rate shifts.'' \emph{Proceedings of the National Academy
of Sciences} 2011 (Mar). doi:10.1073/pnas.1016876108.

%\begin{sloppypar} % more room to wrap urls
%\bibliography{}
%\end{sloppypar}

\end{document}
